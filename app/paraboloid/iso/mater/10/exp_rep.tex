\documentclass[12pt,epsfig]{article}
\usepackage{graphics}
\usepackage{times}
%%%%%%%%%%%%%%%%%%%%%%%%%%%%%% User specified LaTeX commands.
\textheight 24cm
\textwidth 16cm
\tolerance=400
\hoffset=-1.5cm
\voffset=-1.5cm
\language=0

\begin{document}
\noindent
{\bf INTERNATIONAL \hfill     ISO TC 20 / SC 14 / WG 4}\\
{\bf STANDARD \hfill     ISO/CD 22009}
\vspace{1ex}

\strut\hfill {\bf  COMMITTEE DRAFT}
\vspace{2ex}

{\bf\strut\hrulefill}
\vspace{4cm}

\centerline{\Large \bf Space System - Space Environment}

\centerline{\Large \bf (Natural and Artificial)}

\vspace{2ex}

\centerline{\Large \bf Model of Earth's Magnetospheric }
\centerline{\Large \bf Magnetic Field}
\vspace{2ex}
\centerline{\large \bf (Explanatory Report, Revision 8)}
\vspace{1.5cm}
\strut

\centerline{\bf I.I. Alexeev, S.Yu. Bobrovnikov, V.V. Kalegaev, Yu.G. Lyutov, M.I. Panasyuk,}

\centerline{\it Skobeltsyn Institute of Nuclear Physics,}

\centerline{\it Moscow State University , Russia}

\vspace{0.5cm} \centerline{\bf J. M. Quinn,}

\centerline{\it Geomagnetics Group, U. S. Geological Servey MS 966}

\centerline{\it Federal Center, Denver, CO 80225-0046, U. S. A. }

\vspace{6cm}

{\bf\hrulefill}

\vspace{1cm} \centerline{08 June, 2006}

\newpage
\centerline{\large \bf Explanatory Report}

\section{Background}

   The proposed ISO standard is a set of requirements to the  model of the
magnetic field of the Earth's magnetosphere. Such  model is intended  to
calculate the magnetic induction field generated from a variety of current
systems located on the  boundaries and within the boundaries of the Earth's
magnetosphere under a wide range of environmental conditions, quiet and
disturbed, affected by Solar-Terrestrial interactions simulated by Solar
activity such  as Solar Flares and related phenomena which induce terrestrial
magnetic disturbances such as Magnetic Storms. The magnetospheric model 
elaborated on the base of standard will work in
\begin{itemize}
\item Data Analyzing
\item Watching the space environment conditions
\item Forecasting the space environment conditions
\item Warning the expected extremal conditions
\item Postcasting and case study
\end{itemize}


The current (8-th) revision of the Explanatory report is based on
recommendation about the model users and their requirements which described in
Resolution 145  of 16-th ISO TC20/SC14/WG4 Meeting at Houston, USA (October,
2002), Resolution 189  of 20-th ISO TC20/SC14/WG4 Meeting at Las Cruses, USA
(October, 2004) as well as on "Report by IAGA Division V  on a Survey  of
Magnetospheric Modellers  Re-sponses to the Proposed ISO Standard" by Vladimir
Papitashvili and Alan Thomsonas and on the recommendations presented in the
letter from IAGA President, Charles Barton to ISO regarding IAGA position on
the magnetospheric model standardisation (on 5 October 2004).


\subsection{Purposes and scope}


The goals of standardization of the Earth's magnetospheric magnetic
field are:
\begin{itemize}
\item providing the unambiguous presentation of the geomagnetic field of
currents flowing inside the Earth and the magnetic field of magnetospheric
currents;

\item providing comparability of results of interpretation and
analysis of space experiments;

\item providing less labour-consuming character of calculations of
the magnetic field of magnetospheric currents in the space at
geocentric distances of 1-6.6 Earth's radii ($R_E$);

\item providing the most reliable calculations of all elements of
the geomagnetic field in the space environment.
\end{itemize}

The main purposes of elaboration of the "Model of the Earth's
Magnetospheric Magnetic Field" Standard are

\begin{itemize}

\item standardization of method of presentation of the magnetospheric
magnetic field as a sum of magnetic fields
induced by large-scale magnetospheric current systems;
\item standardization of the set of physical parameters of
magnetospheric current systems ("key model parameters");
\item standardization of the methods of magnetospheric magnetic field
calculations by temporal variations of the magnetospheric current systems
parameters;
\item elaboration of the quantitative model of  the Earth's
Magnetospheric Magnetic Field dependent on physical parameters.
\end{itemize}

Technique of calculation of the main parameters of the magnetospheric model
  using data of measurements in the Earth's environment is not the subject of
  standardization.

\subsection{Identification of users}

The model will be useful for user communities working with space related
applications in

\begin{itemize}
\item space weather forecasting;
\item human health and insurance companies;
\item radiation hazards determination;
\item estimation of spacecraft and electronic devices integrity;
\item pure and applied space physics environmental research;
\item sattellite's navigation;
\item satellite's communication;
\item aero and space industry;
\item fundamental scientists;
\item students in space researches.
\end{itemize}

\subsection{Users requirements}

The ultimate objective of space weather research is the development of
quantitative forecasting models. The main requirements of the Space Weather
community to magnetospheric model are

\begin{itemize}
\item Model reliability:

\begin{enumerate}

\item  the model of the magnetospheric currents  magnetic
field must describe a regular part of the magnetic field, its dependence on the
interplanetary medium parameters and reflects such magnetospheric magnetic
field features as depression of the Earth's magnetosphere on the dayside due to
its interaction with the solar wind, day and seasonal variations of the
magnetic field;

\item  the model takes into account the tilt angle between the geomagnetic
   dipole and plane orthogonal to the Earth-Sun line varying within
   a range from -35 to +35 degrees;

\item the model describes dynamics of the magnetospheric large-scale
   current systems;

\item  the model takes into account the dependence of the magnetic field of the
   magnetospheric current systems  on the conditions in the
   Earth's environment;

\item  the model enables taking into account variations of the magnetopause form and
location as well as IMF penetration into the magnetosphere in dependence on
solar wind conditions.
\end{enumerate}

\item  Algorithm simplicity allowing the real-time (near real-time)
calculations.

\item  The model must satisfy the well founded physical principles.

\item The model is working without restrictions imposed on the values
   of interplanetary medium parameters that enable description of the
   disturbed and extremally disturbed magnetosphere for space
   weather forecastings.

\end{itemize}

\subsection{ Specification of the usefulness of the
proposed  standard}

The model is used for instance to
harden both spacecraft and the sensitive electronic devices carried  by them
from the effects of high-energy cosmic radiation. Space Weather Forecasting has
value to  the electrical utility community which can loose of millions of
dollars in equipment such as  transformers and in downtime due to magnetic
storm induction effects at the Earth's surface. For  example, during the March
1989 magnetic storm the Quebec Hydroelectric Utility grid failed  costing on
the order of 8600 million. Other utility grids around the world were also
adversely  affected by this storm. Transcontinental oil and gas pipelines are
also susceptible to  electromagnetic induction due to Space Weather, which
causes corrosion and high-voltage hazards.  Similar problems occur with
spacecraft such as communications and weather satellites. These  satellites are
often in geostationary orbits. During a magnetic storm, the magnetospheric
boundary,  called the magnetopause, which ordinary acts as a deflecting shield
against the Solar Wind is  compressed inside the orbits of these satellites,
leaving them exposed to the full impact of the Solar  Wind's radiation. Sometimes
the orientation of the satellites is determined by sensing the geomagnetic
field. During a magnetic storm, when the magnetopause is compressed inside a
satellite's orbit, the  magnetic field orientation is suddenly reversed causing
the satellite to abruptly rotate, which inturn  causes extended booms (e.g.
gravity gradient stabilization booms and others carrying electronic  sensors)
to snap off, leaving the satellite dysfunctional. The strong increase of the
radiation belts ions as well as relativistic electron fluxes in the inner
magnetosphere during magnetic storm main and recovery phases is the factor
which can also be responsible for the geostationary satellite dysfunction and
even losses. With sufficient warning through Space  Weather forecasting, such
damage can be minimized or eliminated.

Noting that over the past century the degree of Solar activity in terms of the
number of sunspots  occurring during the maximum of the 11-year Solar Cycle has
generally been increasing from one  cycle to the next, with some exceptions,
and noting that the Earth's dipole magnetic field strength,  which accounts for
approximately 90\% of the Earth's total magnetic field, is currently
decreasing  by approximately 7.14\% per century, and further noting that
Earth's dipole field strength has  decreased by 50\% in the past 2000 years,
it is clear that the degree of penetration of energetic  cosmic radiation to
lower altitudes is correspondingly increasing. This in turn heats the upper
atmosphere, which eventually affects the lower atmosphere and subsequently the
daily weather and  storm patterns at Earth's surface as well as posing human
health hazards such as increased cancer  risk due to increased radiation at the
Earth's surface. The more energy penetrating into the Earth's  atmosphere, the
more active atmospheric weather patterns tend to be and the more severe are the
solar  related health hazards likely to be. These facts, plus our increased use
of the space environment,  mean substantially increased hazards related to
Space Weather over the next century. Knowledge  through modeling of the
magnetospheric environment and the resulting ability to predict its behavior
are therefore becoming critical from the points of view of space mission
accomplishment, health,  economics, and natural disaster mitigation. Space
Weather is clearly a global phenomenon, for  which reason it is desirable to have an
internationally accepted magnetospheric model.

\subsection{ Short review of relevant existing models}

At present a lot of the models are used to calculate the magnetic field in the
magnetosphere. The empirical models based on averaging of spacecraft
experimental data (the  OP-74, MF-73, T-87, T-89 models) allowing the real-time
calculations and satisfactory describe the main magnetospheric magnetic field
features during quite times but can not represent the magnetic field dynamics
especially during disturbances. Their parameters are often non-physical and can
not be changed according to rapidly changed magnetospheric conditions.

Versions of the Tsyganenko models, T96 [{\it Tsyganenko}, 1995], and T01 [{\it
Tsyganenko}, 2002], use the observed values of $Dst, {Bz}_{\mathrm{IMF}}$, and
the solar wind dynamic pressure to parameterize the intensity of the
magnetospheric current systems. In the T96 model these parameters are replaced
with the $Kp$ index which has been used in the earlier versions of the
Tsyganenko models.  The T96 (as the earlier Tsyganenko models)  was constructed
using the minimization of the deviation from a data set of the magnetospheric
magnetic field measurements gathered by several spacecraft during many years
(the lage magnetospheric data base by {\it Fairfied}, [1994]. The disturbed
periods are relatively rare events during the observation time, so their
influence on the model coefficients is negligibly small. That is why the T96
model's applicability is limited by $Dst, {Bz}_{\mathrm{IMF}}$, and the solar
wind dynamic pressure low values.

In T01 model the general approach remained the same as in the earlier  T96
model, but the  partial ring current with field-aligned closure currents was
added. The new modeling database included  5-min average B-field data by the
ISTP spacecraft Polar (1996-99) and Geotail (1994-99), as well as by earlier
missions ISEE-2 (1984-87), AMPTE/CCE (1984-88), AMPTE/IRM (1984-86), CRRES
(1990-91), and DE-1 (1984-90).

T04 model [Tsyganenko, Sitnov, 2005] is based mainly on the approaches  used
in T01 model. The main feature of the model is the usage of the disturbed
intervals of the measurements in the constructing of the base data array.

Theoretical models are based on mathematical equations derived from
physical laws and allow calculations of magnetospheric magnetic field for any
level of disturbance. The investigation of the magnetospheric current
systems during magnetic storm is possible in terms of the modern dynamic
models of the magnetospheric magnetic field (OP-88, HV-95, A96). An
important advantage of the paraboloid model A96 [{\it Alexeev et al.},
1996] is that it can describe the magnetic field of each magnetospheric
current system as a function of its own time-dependent input parameters.  A
functional dependence of the model input parameters on the empirical data
obtained by the satellites and on-ground observatories is determined by a
set of submodels. The term ``submodel'' is used for the analytical
definition of each model input parameter as a function of the empirical
data. The last A2000 version of the model [{\it Alexeev et al.},
2001] is presented in the current version of standard as a working example.

The event-oriented models (see for example [{\it Ganushkina et al.,},
2002]) are based on averaging of measurements data during one specific event
and intended for the modeling of the processes during this event.
Event-oriented modeling is needed for quite accurate representation of the
observed magnetic field and can be very useful for validations of the
existing global models. It allows to understand, how do they work for
specific events, what could be improved in
them?                                         


\vspace{1cm}
\section{Model Features}

To be certified the candidate model must satisfy to the requirements and
criteria presented in CD22009, Sec.4 and 5 and include the following features:

\begin{enumerate}
\item  day-side/night-side asymmetry (i.e., compression of the magnetosphere
on the day-side and  extension on the night-side due
to the interaction of the Solar Wind with the internal geomagnetic field);
\item   daily and seasonal variations;
\item  the geomagnetic dipole inclination (tilt angle);
relative to the plane orthogonal  to the Earth-Sun line within a range of
$-35^\circ$ to $+35^\circ$;
\item the close relation with the International
Geomagnetic  Reference Field (IGRF);
\item    computation of the magnetospheric induction field includes:
\begin{enumerate}
\item  the magnetic;
field due to the Chapman-Ferraro currents on the magnetopause  screening the dipole
field;
\item  the magnetic field due to the geotail current system magnetic field
including the closure curren6ts on the magnetopause;
\item   the
magnetic field due to the ring current and partial ring current;
\item  the magnetic field due to Region 1 and 2
field-aligned currents and closure ionospheric currents;
\item   the magnetic field due to the magnetopause currents
screening the ring current and other magnetospheric currents.
\end{enumerate}
\item  model depends on a small set of physical input parameters
(model parameterization), characterizing the magnetospheric current sysytems
intensity and location, such
as geomagnetic dipole tilt angle, distance to the subsolar point of the
magnetosphere as well as the other parameters, which can be different in the
different models;

\item the input
parameters depend on real-time, or near-real-time Empirical Data which can include
solar wind and IMF data as well as
Auroral Oval size and location,
AL and Dst magnetic indices, long-term solar variability indices, sunspot number
etc.;

\item  the model characterizes the
magnetospheric magnetic field under both quiet and  disturbed conditions,
without restrictions/limits  imposed on the values of interplanetary medium
or geomagnetic parameters and indices;
\item although in the framework of the proposed standard, the model is
intended to be used inside the geostationary orbit it must be selfconsistently
enable to provide calculations in the whole magnetosphere.
\end{enumerate}

\vspace{1cm}
\section{The Draft Model}

The proposed candidate magnetic field model of the magnetosphere is intended to
satisfy all of the  requirements set forth in Section 2. Although this standard
is intended to characterize the inner magnetosphere, the model must also merge
smoothly into magnetic field generated in other regions  of Earth's environment
such as the core, the ionosphere, the distant tail, near-magnetopause regions
and interplanetary space. The  International Geomagnetic Reference
Field (IGRF) model describes the magnetic field  generated by the
Earth's core. It is produced and updated by International Association of
Geomagnetism  and Aeronomy (IAGA) every 5 years and is not the part
of the proposed  magnetospheric model. 

The total magnetic field is
calculated as a sum of magnetic fields of internal ($B_1$) and magnetospheric
($B_2$) sources:

\begin{equation} B_M=B_1+B_2 \end{equation}

A model of the magnetic field generated by field-aligned currents, which
connect the magnetosphere to the ionosphere, is also included as part of the
magnetospheric model,  as is a model of the magnetosheath magnetic field which
takes into account the Interplanetary  Magnetic Field (IMF) penetration into
the magnetosphere. The model further includes a set of auxiliary physical
models that characterize the magnetic field associated with various current
systems on the  magnetopause and within the magnetosphere itself. These include
a model for the fields  generated by Chapman-Ferraro currents on the
magnetopause that screen the primarily dipole field  of the Earth characterized
by the IGRF model, a model characterizing fields generated by the  geomagnetic
tail current system, a model characterizing fields generated by the ring
current  system, and a model characterizing fields generated by those
magnetopause currents that screen  the ring current. The overall model is
referred to as the Paraboloid Model A2000 is described in [{\it Alexeev et
al.,} 1996; {\it Alexeev et al.,} 2001; {\it Alexeev and Feldstein,} 2001; {\it
Alexeev et al.,} 2003]. In A2000 magnetospheric magnetic field induction $B_2$
is represented in the form:

$$ B_2=B_{sd}(\psi,R_1)+B_t(\psi,R_1,
R_2,  {\Phi}_{\infty})+B_r(\psi,b_R)+B_{sr}(\psi,R_1,b_R)+
{B}_{fac}(I_\parallel) $$

\noindent where
\begin{itemize}
\item $B_{sd}$   is the magnetic field of Chapman-Ferraro
currents on the magnetopause screening the dipole field;
\item $B_t$ is the
geotail current system magnetic field;
\item $B_r$ is the ring current magnetic
field;
\item $B_{sr}$ is the magnetic field of the magnetopause currents screening
the
ring current;
\item ${B}_{fac}$  is the field of Region 1 field-aligned currents.
\end{itemize}
The magnetospheric magnetic field sources, as represented by A2000 are described
in CD22009 Annex A.

The different magnetospheric magnetic field sources depend on different
input parameters (model parameterization). The full set of the paraboloid model
input parameters includes:
\begin{itemize}
\item   ${\bf \psi}$ - geomagnetic dipole tilt angle
\item  ${\bf R_1}$ - distance to the subsolar point of the
magnetosphere
\item   ${\bf R_2}$ - distance to the inner edge of the geotail current sheet
\item   ${\bf \Phi_\infty}$ -
the tail lobe magnetic flux
\item   ${\bf I_\parallel}$ - the total strength of Region 1 field-aligned
currents
\item ${\bf b_r}$ - the ring current magnetic field at the Earth's center
%\item ${\bf b_{r, asym}},{\lambda_{r, asym}}$ - the partial ring current
%magnetic field at the Earth's center and their longitudinal location
\end{itemize}
.
The magnetospheric magnetic field sources, as represented by A2000 are described
in CD22009 Annex A.

All the input  parameters depend on empirical data from Solar Wind (which can
be taken from the upstream solar wind monitors, today the ACE or Wind
satellites), Auroral  Oval data, and the AL and Dst magnetic indices computed
from various geomagnetic observatories  scattered around the Earth's surface.
So, the model magnetic field sources depend on empirical data via input
parameters (see CD22009 Annex B).  Thus, the paraboloid Model consists of three
basic elements:  Empirical data, Input Parameters, and the Model itself.

To calculate the input parameters the different Submodels may be used. It
should be noted that the submodels are not assumed to be standardization
objects since they are created on the base of physical models. They are subject
of the scientific investigations and can be changed in terms of the calculation
techniques presented in current Committee Draft. It is possible to use the
different models to calculate the input parameters. For example the empirical
models of [{\it Roelof, Sibeck}, JGR, 1993, 98, 21421; {\it Shue et al.}, JGR,
1997, 102, 9497; {\it Kalegaev, Lyutov}, 2000, Adv. Space Res., 25, 1489] can
be used for $R_1$ calculations via solar wind dynamic pressure and IMF $B_z$
value. The model user can choose  its own Submodel which describe some input
parameter dynamics based on his own data set.   Three-level structure of the
model, "experimental data -- the parameters of the magnetospheric current
systems -- magnetospheric field", allows to user to use the different submodels
to provide calculations for the case when the dataset is incomplete. Such
approach allows flexible satisfy the user requirements involving them in the
development of  the model appropriated for their own needs and containing their
own "physics".

Model is dynamic in the sense that it can function in real-time or near
real-time depending on the  availability of the empirical data. As was shown
in the numerous papers (see Sec.4) and as presented in Appendix model functions
through the full range of geomagnetic activity, from  Solar Quiet conditions to
severe magnetic storm conditions, and in the whole magnetosphere. Other
magnetospheric models are commonly limited in their range of applicability with
respect to geomagnetic activity and/or by the region of applicability in space.

\subsection{Demonstration of model}
Demonstration of model and methods' needs and
opportunities for model development are presented in Appendix.

\subsection{The model availability}
The model is available at WWW site
http://www.magnetosphere.ru/iso.

\vspace{1cm}
\section{The Recent Project Activities}

\begin{itemize}

\item Comparisons of the model calculations  with the Large Magnetosphere
Magnetic Field Data Base (Faierfield, et al., Journal of Geophysical Research,
V.11, p.11319-11326, 1994) and cross-comparison with the T06 and T01 models
were completed (see Appendix, Sec. 1).

\item Detailed comparisons of the model calculations with observations in
the course of several magnetic storms were made in the recently published papers:

\begin{itemize}
\item V.V. Kalegaev, I.I. Alexeev, Y.I. Feldstein, L.I. Gromova,
A. Grafe,  M. Greenspan, (in Russian), {\it Geomagn. Aeronom., V. 38}, N 3,
10, 1998.
\item
Dremukhina, L. A., Ya. I. Feldstein, I. I. Alexeev, V. V. Kalegaev  and M.
Greenspan, {\it J. Geophys. Res., 104}, N12, 28,351, 1999.
\item
V.V.Kalegaev, and A. Dmitriev, {\it Advances in Space Research. 26}, N1,
117, 2000.
\item Alexeev, I. I., and Y. I. Feldstein, {\it J. Atmos. Sol. Terr. Phys.,
63/5}, 331, 2001.
\item Kalegaev, V. V., I.I.Alexeev, Ya.I.Feldstein, {\it J.
 Atmos.  Sol.  Terr. Phys.,  63/5}, 473, 2001.
\item Alexeev I.I., E.S.Belenkaya, R. Clauer, {\it Journ. of Geoph. Res.,
105}, 21,119, 2000.
\item R. Clauer, Alexeev I.I., E.S.Belenkaya, {\it Journ. of Geoph. Res.,
 106}, 25695, 2001.
\item Alexeev, I. I., V. V. Kalegaev, E. S. Belenkaya, S. Y.
Bobrovnikov, Ya. I. Feldstein and L. I. Gromova,
{\it Journ. of Geoph. Res., 106}, 25683, 2001.
\item Feldstein, Y.I., L.A. Dremukhina, A.E. Levitin, U. Mall, I.I. Alexeev,   
and V.V. Kalegaev, Energetics of the magnetosphere during the magnetic storm,   
Journ. of Atm. and Sol.-Terr. Phys., V.65, No.4, pp 429-446, 2003.
\item Alexeev I.I., E.S.Belenkaya, S.Y.Bobrovnikov, V.V.Kalegaev, Modelling of
the electromagnetic  field in the interplanetary space and in the Earth's
magnetosphere.Space Sci. Rev., 107, N1/2, 7-26, 2003. 
\item Belenkaya E.S., I.I.Alexeev, and C.R.Clauer, Field-aligned current 
distribution in the transition current system,   Journal of Geophys. Res., 
V.109, A11207, doi:10.1029/2004JA010484, 2004
\item Kalegaev V. V., N. Yu. Ganushkina, T. I. Pulkkinen, M. V. Kubyshkina, H.
J. Singer, and C. T. Russell. Relation between the ring current and the tail
current during magnetic stormsAnn. Geoph., 26, No.2, 523-533, 2005 
\item Alexeev, I.I., What defines the polar cap and auroral oval diameters? in
'The Inner Magnetosphere: Physics and Modeling' ed. Pulkkinen et al., Geophys.
Mon. Ser., V. 155, 257-262, AGU, W., 2005.
\item  Kalegaev V.V., N. Yu. Ganushkina, Global magnetospheric dynamics during
magnetic storms of different intensities. in 'The Inner Magnetosphere: Physics
and Modeling' ed. Pulkkinen et al., Geophys. Mon. Ser., V. 155, 293-301, AGU,
W., 2005.2005.
\item Feldstein, Y.I., A.I. Levitin, J.U. Kozyra, B.T. Tsurutani, A.
Prigancova, L. Alperovich, W.D. Gonzalez, U. Mall, I.I. Alekseev, L.I. Gromova,
and L.A. Dremukhina, Self-consistent modeling of the large-scale distortions in
the geomagnetic field during the 24-27 September 1998 major magnetic storm, J.
Geophys. Res., 110, A11214, doi:10.1029/2004JA010584, 2005
\end{itemize}
\noindent
(see also Appendix, Sec.2).

\item The Drafts of the standard have been sent to
interested
scientists and specialists from many countries and  have been reviewed 
and corrected by the scientists and specialists from

{  \begin{itemize}
\item Daniel M. Boscher (ONERA-CERT/DESP, France)
\item Hiroshi Suzuki, Yukihito Kitazawa (Ishikawajima-Harima
Heavy Industries Co., Ltd.), Prof. T. Iemori (Kyoto University, Japan)
\item Dr. Alexandr Schevurev (STC "Kosmos") and  Dr. V.P. Nikitskiy (RS
Corp. "Energiya", Russia)
\item Tuija Pulkkinen, Natasha Ganushkina (Finnish Meteorological
Institute, Finland)
\item Charles Burton, Vladimir Papitashvili, Alan Tomson (IAGA)
\end{itemize}}

\item The mmodel joint testing was made in collaboration with scientists from
different labs in Finland, Russia, Sweden, USA. 

\item The Web site of the Project has been created
(http://www.magnetosphere.ru/iso/)


\item The project "SPACE ENVIRONMENT  (NATURAL AND ARTIFICIAL). MODEL OF THE
EARTH'S MAGNETOSPHERIC MAGNETIC FIELD." has been discussed and approved at the
recent Meetings of WG4.

\begin{itemize}
\item In accordance with resolution, taken at the plenary meeting of
TC20/SC14/WG4 in Brazil, and with request made by James E. French (Secretary of
ISO TC20/SC14), the Approved Work Item (AWI) "Space systems - Space
environment - Model of the Earth's magnetospheric magnetic field" has been
registered with the ISO Central Secretariat and given the number 22009.
\item In accordance with resolution 122, taken at the plenary meeting of
TC20/SC14/WG4 in Tolouse (October, 2001) the current draft
standard was examined and revised to incorporate new content and
format as described in resolution 114 of the Meeting.
\item At the plenary meeting of
TC20/SC14/WG4 in Noordwijk, Netherlands (May, 2002) the experts of WG4 confirm
the necessity of involving th IAGA experts in the comparative analysis of
existing magnetic field models.
\item In accordance with resolution 145 approved by 16-th ISO TC20/SC14/WG4 
Meeting at Houston, USA (October, 2002) WD22009 was reexamined and revised  
to incorporate new recommendation about the model users and their requirements 
obtained from IAGA representatives.
\item In accordance with resolution 161 approved by 17-th ISO TC20/SC14/WG4 
Meeting atTsukuba, Japan in May 2003 WD22009 was adopted as a Committee Draft 
CD22009  
\item In  resolution 189 approved by 20-th ISO TC20/SC14/WG4 
Meeting at Las Cruses, USA (October, 2004) the experts of WG4 recommend:

1. to continue the Standard elaboration guided by IAGA recommendations and using the ISO
comment sheet;

2. to renew the work on extending the list of the candidate models;

3. to establish contact with IAGA experts and to discuss with IAGA a possibility to
organize a special symposium in the context of IAGA General Assembly in 2005 or 2007
and devote it to the questions of the magnetospheric model, IGRF and magnetospheric
standard model interconnections.

\end{itemize}

\item The new comparison of the model calculations with experimental data as
well as the models cross-comparisons was made. The results of comparison are
published in the international journals (more 20 refereed articles from
1998).  

\end{itemize}

\vspace{1cm}
\section {Contacts with IAGA}

In 2001, IAGA was invited by Dr. John Quinn (then the U.S. Representative to
the ISO TC20/SC14 Working Group 4 "Space Environment, Natural and Artificial")
to assist in the evaluation of the proposed magnetospheric standard.
Discussions were held at business meetings during IAGA's 2001 Scientific
Assembly in Vietnam and a small task force of IAGA scientists was set up to
participate in the WG4 meetings in Toulouse, France (October 2001), Noordwijk,
Holland (May 2002), and Houston, USA (October 2002). The Task Force undertook a
survey of the magnetospheric model community and reported to the IAGA Executive
Committee at the IUGG General Assembly in Sapporo, Japan in July 2003.
Conclusions and opinions were forwarded to Dr. W. Kent Tobiska, WG4 Secretary
and to the authors of the proposed magnetospheric standard. 

In accordance with the letter from IAGA President Charls Burton from 5 October
2004 to Subcommittee 14 "Space systems and operations" ISO Technical Committee
20 IAGA recommends a process-based approach to ISO standards in the space
environment area, namely that standards be formulated as a set of
specifications that must be met - for solar irradiance, radiations in geospace,
magnetospheric models, and so on. Producers of models must demonstrate
compliance with the standard, which is adaptive to advances in knowledge and
changes in user requirements. We believe this approach avoids the unwelcome
consequences raised above. We note that the solar irradiance and atmospheric
density groups within WG4 have adopted process-based specification of
standards. 

Detailed responses to IAGA recommendations were sent to IAGA. In accordance
with IAGA recommendation ISO CD 22009 was sufficiently reconsidered: it was
declared that IGRF is not a subject of standardization in terms of CD22009;
authors agreed to consider Standard 22009 as a set of specifications and
requirements; Paraboloid Model was included to the standard in supplement as a
�working example� satisfying all the criteria and requirements.

MSU working group agreed with the comments regarding the new model checking on
the real data and ready for independent testing and cross comparison of the
model with the other candidate models (if any).  However, this  activity must
be carried out within the existing already Project timetable  accepted and
approved by ISO TC20/SC14/WG4 and without the destroying the existing ISO
procedure of the Standard development.

The new approach and recent progress in the standard development �Model of
Earth�s magnetospheric magnetic field. International standard (ISO/CD 22009)�
was presented in oral talk on GAV06 Symposium �Uses and applications of
geomagnetic field models� during IAGA General Assembly in  Toulouse (2005). 


\newpage
\noindent
{\bf APPENDIX. Accuracy of the Model and Comparison
With Experimental
Data}
%\end{document}

In this section we show the examples of paraboloid model usage and comparison
with the other models. There are the comparison with the
Large Magnetosphere Magnetic Field Data Base (subsection A1), calculation
of the magnetic field during January 1997 magnetic storm, calculation of the
magnetic field during September 1998 magnetic storm using the advanced submodels
and the cross-comparison of the paraboloid model, T01 model and event-oriented
model by  [{\it Ganushkina et al.}, 2002, 2003] for the 25-26 June 1998 case
study.


\vspace{0.7 cm} {\bf A1. Stationary Case. Comparison with the
Large Magnetosphere Magnetic Field Data Base}

\begin{figure}
{\par\centering
\resizebox*{8cm}{!}{\includegraphics{./fig/gist.eps}}}
\resizebox*{7cm}{!}{\includegraphics{./fign/scat1.eps}}

\caption{\baselineskip=0.8ex {\small The distribution of discrepancies of
magnetospheric field magnetic induction calculated in term of the paraboloid
model, compared with experimental data and Scatter plot of the observed magnetic field
module against calculated ones. \protect\( B(sc)\protect \),nT, is
the field measured onboard spacecrafts, \protect\( B(mod)\protect \),nT, is
the field calculated in terms of the parabolic model, N is for
statistics.} }
\end{figure}

The comparison with the Large Magnetosphere Magnetic Field Data Base
(Faierfield et al., Journal of Geophysical Research, V.99, p.11319-11326, 1994)
was made. Data base includes data of Explorer 33,35, IMPs 4,5,6,7,8, Heos 1,2
and ISEE 1, 2 in the region between 4 and 60 \( R_{E} \). Calculations of the
input parameters of the model were performed using in terms of submodels
presented in Annex B of CD 22009. 

\begin{figure}
{\par\centering
\resizebox*{12cm}{!}{\includegraphics{./fig/map.eps}} \par }

\caption{\baselineskip=0.8ex {\small The distribution of the discrepancies of
magnetospheric magnetic field calculated in term of the paraboloid model A2000,
compared with experimental data. In the singled out cell  the format of data is
shown: \protect\( B(sc)\protect \),nT, is the measured onboard  spacecrafts
magnetic field averaged in cells, \protect\( B(mod)\protect \),nT, is the
average field calculated in terms of the paraboloid model, N is for
statistics.}\small }
\end{figure}

\begin{figure}

{\par\centering \resizebox*{14cm}{!}
{\rotatebox{270}{\includegraphics{./fig/map99.eps}}} \par}
\caption{\baselineskip=0.8ex {\small The magnetic field module distributions
in the different cells of the
magnetosphere, measured (solid line) and calculated by paraboloid model A2000
(thin line).}}
\end{figure}

Fig.1 and 2 display this comparison in the form of the distribution of
discrepancies and in the form of scatter plot (in the inner magnetosphere). 
The histogram in Fig.1 shows the distribution of relative
discrepancies  $D=\frac{B(sc)-B(mod)}{B(sc)}\cdot 100\%$ integral over the
whole experimental material (45181 measurements). The discrepancy mean value is
about +3\% (the distribution is asymmetric with a long negative
\char`\"{}tail\char`\"{}), \( \sigma  \) of  the distribution being of \(
\sim 80\% \). Fig. 2 presents the distribution of absolute and relative
discrepancies differential in \( x \) and  \( \rho  \), where  \(
\rho=\sqrt{y^2+z^2}  \), $x,y,z$ are the solar-magnetospheric (GSM)
coordinates. The weight of each discrepancy value (statistics) is shown in the
corresponding cell in \( x \) and \( \rho  \). An examination of the Fig.2
shows that near the Earth at distances about the geostationary orbit in the
magnetosphere nightside the discrepancy is, on average, 12.3 nT for \(
-10<x\leq 0 \) and \( 0\leq \rho <10 \), and in the magnetosphere dayside it
is, on average, 3.4 nT for \( 0<x\leq 10 \) and \(0\leq \rho <10 \).

The magnetic field module distributions in the different cells of the
magnetosphere, measured and calculated by paraboloid model are represented in
the Figure 3. The measured magnetic field has also an own non-Gaussian
distribution in each cell. The mean values and discrepancies represented in the
Fig. 2 are the measured magnetic field distribution mean values and mean
discrepancies between measured and calculated values. The calculated by
paraboloid model magnetic field is distributed in a good agreement with
observations.

\begin{figure}
{\par\centering
\resizebox*{4cm}{!}{\includegraphics{./fig/b1_a99.eps}}
\resizebox*{4cm}{!}{\includegraphics{./fig/b2_a99.eps}}
\resizebox*{4cm}{!}{\includegraphics{./fig/b3_a99.eps}}
\par}
\caption{\baselineskip=0.8ex {\small The magnetic field module distributions
(measured and calculated by A99) in the near-Earth $(x,\rho)$ cells (10:0;
0:10), (0:-10; 0:10), (-10:-20; 0:10) of the
magnetosphere.}}
\end{figure}

\vspace{0.3cm}
\begin{figure}
{\par\centering
%\resizebox*{4cm}{!}{\includegraphics{./fig/bx1_a99.eps}}
%\resizebox*{4cm}{!}{\includegraphics{./fig/bx2_a99.eps}}
\resizebox*{4cm}{!}{\includegraphics{./fig/bx3_a99.eps}}
\par}
\caption{\baselineskip=0.8ex {\small The same as in the Fig. 4 for GSM $Bx$
component of the magnetospheric magnetic field.}}
\end{figure}


Fig.4 represents the distributions in $(x, \rho)$ cells  (10:0; 0:10), (0:-10;
0:10), (-10:-20; 0:10) respectively. The first and second cells demonstrate the
regular shifts: the magnetic field in the night side is underestimated and in
the dayside is overestimated. Such behavior can be explained by field aligned
currents effect which is not taken into account in these calculations. The
quite good agreement exists in the third cell.

Fig. 5 represents the same comparison for the GSM Bx component of the magnetic
field  in the $(x, \rho)$ cell (-10:-20; 0:10). The distribution depression
near the $Bx=0$ corresponds the measurements which were made in the tail plasma
sheet  region. The A99 paraboloid model has infinite thin tail current so the
$Bx$ values which are about of zero are absent.

\begin{figure}
{\par\centering \resizebox*{12cm}{!}
{\includegraphics{./fig/map01rms.eps}} \par}
\caption{\baselineskip=0.8ex {\small The same as in the Fig. 3, calculated in
terms of A01 and T96 models}}
\end{figure}

Paraboloid model allows flexible taking into account the new magnetospheric
magnetic field sources. Moreover, because each magnetospheric magnetic field
source with its own screening currents is calculated separately and depends
linearly on its own input parameters we can change the parametrization of
current systems to match better the data. To take into account the mentioned
above effects of field aligned currents and "thin" tail current the new "beta"
version of paraboloid model (A01) was developed.  The "thin geotail" magnetic
field [{\it Alexeev and Bobrovnikov, 1997}] and field aligned current magnetic
field [{\it Alexeev, Belenkaya and Clauer, 2001}] were taken into account.
Fig. 6 shows
the measured magnetic field module mean values in the different cells of the
magnetosphere as well as mean discrepancies between the measured magnetic field
and calculated by A01 (second row) and T96 (third row). The more good agreement
for A01 is detected in the near-Earth region than that represented in the
Figure 2. We can see that in general, the obtained in terms of A01/A99
discrepancies are of the same order as those obtained in the framework of T96
model [{\it Tsyganenko,} 1995].


\begin{figure}
{\par\centering \resizebox*{12cm}{!}{\rotatebox{270}
{\includegraphics{./fig/map01.eps}}} \par}
\caption{\baselineskip=0.8ex {\small The magnetic field module distributions 
over
the whole statistics in the different cells in the Earth's magnetosphere,
measured and calculated by A01 model.}}
\end{figure}

\newpage

\begin{figure}
{\par\centering
\resizebox*{4cm}{!}{\includegraphics{./fig/b1_a01.eps}}
\resizebox*{4cm}{!}{\includegraphics{./fig/b2_a01.eps}}
\resizebox*{4cm}{!}{\includegraphics{./fig/b3_a01.eps}}
\par}
\caption{\baselineskip=0.8ex {\small The same as in the fig. 4 for magnetic field
module calculated by A01 model.}}
\end{figure}


\vspace{0.3cm}
\begin{figure}
{\par\centering
%\resizebox*{4cm}{!}{\includegraphics{./fig/bx1_a01.eps}}
%\resizebox*{4cm}{!}{\includegraphics{./fig/bx2_a01.eps}}
\resizebox*{4cm}{!}{\includegraphics{./fig/bx3_a01.eps}}
\par}
\caption{\baselineskip=0.8ex {\small The same as in the fig. 8 for $Bx$ component
of the magnetospheric magnetic field.}}
\end{figure}
\vspace{0.3cm}

The magnetic field module distributions in the different cells of the
magnetosphere, measured and calculated by paraboloid model are represented in
the Figure 7. The magnetic field distributions measured in the near-Earth's
cells represented in the Figures 8 (magnetic field module) and 9 ($Bx$
component). The distributions of the magnetic field calculated for the
different parameterizations of the tail current and field aligned currents
demonstrate the more good agreement with experimental data of calculations obtained
in the framework of A01 paraboloid model than that obtained by A99 model.

The results represented on the Fig. 6 show that paraboloid model, analytical
and based on the small number of the input parameters, describes the large
array of experimental data with approximately the same accuracy as the T96
model, which constructed as approximation of that array. Fig. 10  represents the
distributions of measured and calculated by A99 and T96 models magnetic fields
(module and $Bx$ component, respectively).

\vspace{0.3cm}
\begin{figure}
{\par\centering
\resizebox*{6cm}{!}{\includegraphics{./fig/gistbal.eps}}
\resizebox*{6cm}{!}{\includegraphics{./fig/gistbxal.eps}}
\par}
\caption{\baselineskip=0.8ex {\small The same as in the fig. 8 for $Bx$ component
of the magnetospheric magnetic field.}}
\end{figure}
\vspace{0.3cm}


%=================================================8/06/2006
In the Table 1 the comparison of magnetic field calculated by paraboloid model
(A99) Tsyganenko model (T96) and measured magnetic field from Large
Magnetosphere Magnetic Field Data Base averaged by the levels of disturbances
is presented. We can see that only for very quite conditions T96 model gives
the better results than A99. For \( Kp \) between \( 1^{-} \) and \( 2^{-} \)
the results are comparable, but for disturbed conditions (\( Kp>2 \)) A99 gives
the better results than T96. The T96 (as the earlier Tsyganenko models)  was
constructed using the minimization of the deviation from a data set of  the
magnetospheric magnetic field measurements gathered by several spacecrafts
during many years. The disturbed periods are relatively rare events during the
observation time, so their influence on the  model coefficients is negligibly
small. That is why the T96 model's  applicability is limited by $Dst,
{Bz}_{\mathrm{IMF}}$, and the solar wind dynamic pressure low values.

{ \centering
\label{at1} \medskip

\begin{table}
\begin{tabular}{|l|r|r|r|}
\hline
\( Kp \)&
A99 &
T96 &
Data\\
\hline
\( 0,0^{+} \)&
13.8 &
{\bf 14.9} &
15.5 \\
 \( 1^{-},1 \)&
{\bf 16.9} &
16.3 &
17.6 \\
 \( 1^{+},2^{-} \)&
18.3 &
{\bf 18.6} &
20.3 \\
 \( 2,2^{+} \)&
{\bf 21.6} &
20.6 &
22.6 \\
 \( 3^{-},3,3^{+} \)&
{\bf 25.3} &
24.1 &
26.3 \\
 \( 4^{-},4,4^{+} \)&
{\bf 30.0} &
28.1 &
31.3 \\
 \( 5^{-},5 \)&
{\bf34.8} &
33.4 &
35.4  \\
\hline
\end{tabular}

\caption{\baselineskip=0.8ex {\small Comparison of magnetic field calculated by paraboloid
model (A99) Tsyganenko model (T96) and measured magnetic field from Large Magnetosphere
Magnetic Field Data Base averaged by the levels of disturbances.}\small }
\end{table}
}
%\end{document}

\newpage

\vspace{0.2cm}
\textbf{A2. Nonstationary Case. Case study for January 9--12, 1997}

The dynamics of the magnetospheric current systems was studied in [{\it Alexeev
et al.,} 2001] in terms of A99 model for
the specific magnetospheric disturbance on January 9--12, 1997 caused by the
interaction of the Earth's magnetosphere with a dense  solar wind plasma
cloud.  A dense cloud of the solar wind plasma was of rather complicated
structure.  A southward interplanetary magnetic field (IMF) in its leading part
caused a significant substorm activity during the interaction with the
magnetosphere. A strong increase of the relativistic electron fluxes at the
geosynchronous orbit was observed [{\it Reeves et al.,} 1998]. The trailing
half of the magnetic cloud contained a strong northward IMF and was accompanied
by a large density enhancement that strongly compressed the magnetosphere.
Because of the significant compression of the magnetosphere, several
magnetopause crossings by the geostationary orbit took place. This storm  causes
also the crash of geostationary satellite Telstar 401 leading to significant
financial losses.

Figure {\ref{F1}} shows the $Dst$ and $AL$ indices (Figures
{\ref{F1}}a and {\ref{F1}}b). The hourly averaged Wind data on the
plasma and magnetic field are presented in Figures~{\ref{F1}}c--{\ref{F1}}e.
The time delay ($\sim 25$~min) between the
measurements in the Earth's vicinity and on  board the Wind spacecraft
is taken into account.

\begin{figure}
{\par\centering \resizebox*{6cm}{!}{\includegraphics{./fig/janf1.eps}} \par }
%\mbox{\epsfbox{./fig/f1.eps} }
\caption {\baselineskip=0.8ex {\small  Empirical data used for the
calculations of the model input parameters:
(a) $Dst$,
(b) $AL$,
(c) IMF $B_z$ component of
the solar wind,
(d) velocity, and
(e) density
for January
9--12, 1997. }\small}
\label{F1}
\end{figure}

The model input parameters
were defined by the solar wind density and velocity, by the strength and
direction of the interplanetary magnetic field, and by the auroral $AL$ index.
Figure~{\ref{F2}} presents the time variations of the model input parameters:
the tilt angle (Figure~{\ref{F2}}a) and the magnetic field flux across the
magnetotail lobes (Figure~{\ref{F2}}b). Figure~{\ref{F2}}c shows $b_r$,
calculated  using Burton equation and Dessler-Parker-Skopke equation.
Figure~{\ref{F2}}d shows the distances to the magnetopause subsolar point  and
to the earthward edge of the tail current sheet.

\begin{figure}[thb]
{\par\centering \resizebox*{6cm}{!}
{\includegraphics{./fig/janf2.eps}} \par}
\caption { \baselineskip=0.8ex {\small  The model
input parameters for January 9--11, 1997: (a) the tilt
angle, $\psi$; (b) the magnetic field flux through the magnetotail lobes,
$\Phi_{\infty}$; (c) the ring current magnetic field at the Earth's
center $b_r$; and (d) the distances to the magnetopause subsolar
point (solid curve), $R_1$, and to the earthward edge of the magnetotail
current
sheet
(dashed curve), $R_2$.}}
\label{F2}
\end{figure}

To investigate the $Dst$ sources during the January 9--12, 1997, event the
ground magnetic field was analyzed  in terms of the paraboloid model of the
magnetosphere A99, which allows us to distinguish the contributions of
different large-scale current systems. The paraboloid model calculations are
demonstrated in Figure~{\ref{F4}} at the left panel. The magnetospheric
magnetic field  variation is calculated at the geomagnetic equator at each hour
of  magnetic local time (MLT) and averaged over the equator.
Figures~{\ref{F4}a--\ref{F4}c} (left panel) present the $Dst$ sources $Bcf$,
$Br$, and $Bt$  and their parts arising owing to the Earth currents.
Figure~{\ref{F4}}d compares the $Dst$  and the calculated magnetic field. A
good agreement is obtained for both the relatively quiet and disturbed periods.
The calculations in terms of the paraboloid model give an RMS deviation from
$Dst$ ($\delta B$) of $\sim8.7$~nT.
\vspace{0.3cm}
\begin{figure}
{\par\centering
\resizebox*{6cm}{!}{\includegraphics{./fig/janf4.eps}}
\resizebox*{6cm}{!}{\includegraphics{./fig/t96.eps}}
\par}
\caption{\baselineskip=0.8ex {\small
{\bf Left:}
(a) Magnetic field of currents on the magnetopause, (b, c) the ring current
magnetic field and  tail current magnetic field, respectively, at the Earth's
surface (solid curves) and the corresponding magnetic field  due to currents
induced inside the Earth (dashed curves), and (d) $Dst$ (heavy solid curve) and
total magnetic field, $B_{M}$ (dashed curve), calculated at the Earth's surface
by A99 in the course of the magnetic storm on January 9--12, 1997.
{\bf Right:}
$Dst$ (heavy solid curve) and total magnetic field, $B_{M}$ (dashed
curve), calculated at the Earth's surface by T96 in the course of the magnetic
storm on January 9--12, 1997 [{\it Turner et al.}, 2000].
}}
\label{F4}
\end{figure}
\vspace{0.3cm}



We can see from the analysis of magnetic storm on
January 9--12, 1997, that the magnetospheric dynamics depends on all the
magnetospheric magnetic field sources, which appear to be comparable by the
order of magnitude. The paraboloid model can be successfully applied,
especially in the disturbed periods, when the empirical models are often not
valid.

The same magnetic storm was investigated by [Turner et al., 2000] in terms of
T96 model. The important feature of the T96 model is (as reported by the author
in the T96\_01 model's description) its applicability only for $20\;
\mbox{nT}>\,Dst\,>-100$~nT, $0.5$~nPa$<p_{sw}<10$~nPa, and
$-10$~nT$<Bz_{IMF}<10$~nT. In the course of the storm under consideration
(January 9-12, 1997) the upper value of $p_{sw}$ is significantly beyond the
$10$~nPa limit. During the most disturbed interval of the magnetic storm under
consideration (the first hours of 11 January 1997) T96 model was out of order
(see Figure~{\ref{F4}}, right panel).

The Figure~{\ref{F4}} right panel represents calculations made by T96 model
in [Turner et al., 2000].
The reason for the residual difference between the calculations presented in
[{\it Alexeev et al.}, 2001] and those made by {\it Turner et al.} [2000] was
investigated in [{\it Alexeev et al.}, 2001]. This is the tail current inner
edge dynamics which are taken into account in the paraboloid model in
accordance with the auroral oval expansion due to the substorm activity. In the
calculations made by {\it Turner et al.} [2000] the dynamics of the inner edge
of the tail current sheet are neglected.

\begin{figure}
{\par\centering \resizebox*{6cm}{!}
{\includegraphics{./fig/janf3.eps}} \par}
\caption {\baselineskip=0.8ex {\small (a) Comparison of the polar cap radius
calculated from the magnetic
flux value
$\Phi_\infty$ (solid curve) with radii obtained from the measurements
on board DMSP F10-F13 (marked with triangles) and from the Polar Ultraviolet
Imager (UVI) images
(marked with circles). (b) Comparison of the midnight latitude of the
equatorward boundary of the polar oval calculated in terms of paraboloid model
(solid curve) and that calculated by the data measurements
on board  DMSP F10-F13  (marked with triangles).}}
\label{F3}
\end{figure}

Thus the discrepancy of the results obtained in [{\it Alexeev et al.}, 2001]
and in that of {\it Turner et al.} [2000] is explained mainly by the use of
different quantitative models and associated with the difference of the tail
current parameterization. The quality of a model and its flexibility are
defined by the possibility of reflecting the dynamics of the large-scale
current systems. The empirical models do not yet allow one to determine
correctly the time dependence of each large-scale current system. In the
paraboloid model the submodels are used for the calculation of the parameters
of the large-scale magnetospheric current systems. These submodels can take
into account the significant features of various magnetospheric current
systems.

The analysis of the magnetic disturbances during the January 9--12, 1997,
event shows that in the course of the main phase of the magnetic storm the
contribution of the ring current, the currents on the magnetopause, and the
currents in the magnetotail are approximately equal to each other by an order
of magnitude. Nevertheless, in some periods one of the current systems becomes
dominant.  For example, an intense $Dst$ positive enhancement (up to +50~nT) in
the course of the magnetic storm recovery phase in the first hours on January
11, 1997, is associated with a significant increase of the currents on the
magnetopause, while the ring current and the magnetotail current remain at a
quiet level. Such analysis can be made only in terms of the modern dynamical
models such us   paraboloid models, where the different magnetic field sources
can be calculated separately. A comparison of the calculated $Dst$ variation
with measurements indicates  good agreement.

This analysis allows us to investigate the level of applicability of the
different kinds of  magnetospheric models.  The T96 model is not applicable for
disturbed periods and does not take into account the time dependence of the
important parameters of the magnetospheric current systems. For this reason the
most essential part of the magnetotail current system was excluded from the
consideration made by {\it Turner et al.} [2000]. The paraboloid model depends
on the parameters of magnetospheric origin and takes into account the movements
of the magnetotail in accordance with the level of geomagnetic activity.

To estimate the accuracy of  our model calculations of the magnetospheric field
at geosynchronous orbit, a comparison with the data obtained on board  the
geostationary satellites GOES 8 and 9 was performed.  For the verification
of calculations of the magnetotail current contribution to $Dst$, the
obtained values of the model parameters were used to calculate the auroral
oval boundaries,  which were  compared  to the  boundaries obtained using
the DMSP precipitation data and the Polar UVI images.


Figure~{\ref{F3}}a compares the polar cap radius calculated by
paraboloid model to the radii obtained from the observations on  board
DMSP F10-F13  and on  board  Polar. Figure~\ref{F3}a
shows  good agreement between the calculations and the experimental
data obtained from the independent sources. So, the model estimation of
$\Phi_\infty$ can be used to identify the polar cap boundaries.
Figure~\ref{F3}b compares the midnight latitude of the equatorward
boundary of the auroral oval calculated by paraboloid model to those
determined using the particle spectra measured
on  board  the DMSP F10-F13 satellites.  The
obtained agreement with observations confirms our suggestions about
$\Phi_\infty$ and $R_2$ made above.

\begin{figure}
{\par\centering \resizebox*{6cm}{!}
{\includegraphics{./fig/janf6.eps}} \par}
\caption {\baselineskip=0.8ex {\small  Comparison of the magnetic fields
calculated in terms of the
paraboloid model and measured during the magnetic storm on January 9--12 , 1997,
along the (a) GOES 9 orbit and (b) GOES 8 orbit.}}
\label{F6}
\end{figure}

Figure~\ref{F6} presents the calculations of the magnetic field along the GOES
9 and 8 spacecraft orbits. To take into account the magnetic field of the
interterrestrial sources,  the International Geomagnetic Reference
Field (IGRF95) model was used.  The agreement of calculations with the measured
magnetic
field confirms the initial assumptions of the relative roles of the
magnetospheric current systems in the course of magnetic storm.

The usage of the paraboloid model allows one to make an important physical
conclusions about the development of the different magnetospheric magnetic
field sources during disturbances. We can see that during the main phase of a
weak magnetic storm the magnetotail current and the ring current create
disturbances of approximately equal intensities.
The paraboloid model describes well the magnetic field variations on the
Earth's surface and at the geosynchronous orbit during the interaction
of a solar wind plasma cloud with the magnetosphere on January 9--12,
1997. The root mean square deviation between the model calculations
and the measured field is equal to $8.7$~nT. The tail current contribution
to the storm maximum disturbance is about $-60$~nT (for the $Dst$ maximum equal
to $-78$~nT).

\newpage

\vspace{0.2cm}
\textbf{A3. Nonstationary Case. Magnetic Storm Study on September 24 - 26,
 1998}
%\subsection*{Magnetic Storm Study: September 24 - 26, 1998}

In this sectionwe will demonstrate the advanced technique for magnetic
storm investigation which is based on the usage of A01 model and modyfied
submodels for calculations of the input parameters.

The magnetic storm on 24-27 September, 1998 will be  investigated
(see Figure~\ref{fig_data}). An arrival of a dense cloud of the solar
wind plasma at 23:45 on 24 September was accompanied with a northward turning
of the IMF, this direction remains for 3 hours. This leads to the interesting
phenomena in the Earth's magnetosphere, e.g., to a significant decrease
of the polar cap during 30 minutes {\it Clauer et al.} [2001].
After that the IMF had a strong negative north-south component. Moreover,
in 6 hours the second shock wave of the solar wind dense plasma encounters the
magnetosphere. At almost the same time a significant substorm activity has
been detected ($AL$ index). The both events, however, weakly influenced the dynamics
of the \( D_{st} \) variation. This behavior of the \( D_{st} \) index
again attracts attention to the question about dynamics of the magnetospheric current
systems and their relative contributions to the magnetic field at the
Earth's surface.
\begin{figure}
\centerline{%
\begin{tabular}{ll}
\resizebox*{5.3cm}{!}{\includegraphics{fign/figure6l.eps}}&
\hspace*{2mm}\resizebox*{5.3cm}{!}{\includegraphics{fign/figure6r.eps}}
\end{tabular}}
\caption{Solar wind data, $D_{st}$,  $AL$ (left panel), and
the model parameters $R_1,\;R_2$, both in $R_E$, $b_r$ [nT],
\( \Phi _{pc}= \Phi_\infty \) [MWb], and $\Psi\;[^\circ ]$  (right).}
\label{fig_data}
\end{figure}

%\section{Model, Submodels, Calculation of the parameters}

In the paraboloid model the magnetic field and \( D_{st} \) index are
calculated in two stages. Various current
systems depend on the concrete parameters unambiguously and a fixed set of these
parameters unambiguously defines the magnetic field over the entire
magnetosphere.

Calculation of the model parameters in the course of the magnetic storm
under consideration made using submodels described below yielded the result
presented in right panel of Figure~\ref{fig_data}.
The new submodels
other than those used in January 1997 magnetic storm investigation (See A3)
were used to provide more accurate calculations.  Calculation of the first
two parameters are the less contradictory part of this study. The
magnetopause subsolar distance, $R_1$, has been calculated from the
pressure balance at the subsolar point. The calculation scheme is based on
the iterative procedure.

In order to find the distance to the inner edge of the tail
current sheet we have used the results by {\it Feldstein et al.} [1999]
 study:
\begin{equation}
R_{2}=1/\cos ^{2}\varphi _{n}\, , \quad%
 \varphi _{n}=64.9^{\circ }+(|D_{st}|[\mbox{nT}]/31)^{\circ }\, .
\label{Er2}
\end{equation}

%\subsubsection*{Submodels: Ring current}

The calculation of the ring current contribution to the \(
D_{st} \) index is based on the results of {\it Burton et al.} [1975],
{\it O'Brien and
McPheron} [2000] and {\it Alexeev et al.,} [2001].
Starting from the Dessler-Parker-Sckopke relation between $b_r$ and
the ring current particle energy
$\varepsilon _{r}$ {\it Dessler and Parker}
[1959] as well as  from the injection equation
for the dependence of the energy, $\varepsilon _{r}$, on the time:
\begin{equation}
b_{r}=-\frac{2}{3}B_{0}\frac{\varepsilon _{r}}{\varepsilon _{d}}
\, , \quad \mbox{here }\varepsilon _{d}=\frac{1}{3}B_{0}M_{E}\, ,\quad \mbox{and }
\frac{d\varepsilon _{r}}{dt}=U-\frac{\varepsilon _{r}}{\tau }\, ,
\label{eq13}
\end{equation}
we received equation for the ring current field:
\begin{equation}
\frac{db_{r}}{dt}=F(E)-\frac{b_{r}}{\tau }\, , \quad
\tau \left( hours\right) =2.37e^{9.74/\left( 4.78+E_{y}\right) } .
\label{eq17}
\end{equation}
Here injection function, $F(E)$, is determined by the dawn-dusk solar wind electric field $E_{y}$:
\begin{equation}
\label{eq14}
F(E)=f_{pr}\left( p_{sw}\right) f_{ar}\left( AL\right) \left\{ \begin{array}{ll}
d\cdot (E_{y}-0.5) & E_{y}>0.5mV/m .\\
0 & E_{y}<0.5mV/m
\end{array}\right.
\end{equation}

The additional factors \( f_{pr}\left( p_{sw}\right)  \) and \( f_{ar}\left( AL\right)  \)
makes it feasible to take into account the influence on the injection
of the solar wind dynamic
pressure and substorm activity.  Large
values of the solar wind dynamical pressure increase the effectiveness
of injection to the ring current due to increasing of the solar wind plasma transport across
magnetopause and into the plasma sheet. \( AL \) index shows a fraction of energy flowing
directly to the ionosphere. Large values of the \( AL \) index or, on the other
words, increasing of substorm activity
show, from our point of view, that the most part of the energy
accumulated in the magnetotail is directly transferred to the ionosphere.
That's why the injection to the ring current decreases. These effects are described
by factors \( f_{pr}\left( p_{sw}\right)  \), \( f_{ar}\left( AL\right)  \) in Eq.~(\ref{eq14}).

%\subsubsection*{Submodels: Tail current system}

Using an enormous experience of modelling the magnetic storms and
an ideas
developed in the present paper we elaborated new submodel describing the
dependence of the magnetotail current system on the measurement data.
As in the earlier papers (see, e.g., {\it Alexeev et al.,} [2001]), the
magnetic flux in  the magnetotail lobes is the key parameter for both
the tail current system and the magnetosphere as a whole, being calculated
as a sum of two terms:
\begin{equation}
\label{eq4}
\Phi _{\infty }\left( t\right) =\Phi _{0}\left( t\right) +
\Phi _{s}\left( t\right) .
\end{equation}
Here \( \Phi _{0}\left( t\right)  \) is the quiet time value of the magnetic flux,
\( \Phi _{s}\left( t\right)  \) is the magnetic flux value associated with
the electric field enhancement in the solar wind. To calculate the quiet value, use
is made here of our assumption of the basic state of the magnetosphere.

\begin{figure}
{\par\centering \resizebox*{8cm}{!}{\includegraphics{fign/figure7.eps}} \par}
\caption{Model magnetic field (dot-dot-dashed curve) vs
\protect\( D_{st}\protect \) (solid curve). Contributions of
magnetopause currents (dashed curve), ring current (dotted curve), and
tail current
(dot-dashed curve) are demonstrated. One vertical step is $10$~nT.}
\label{fig_dts1}
\end{figure}

For each value of the solar wind dynamical pressure,
the current system parameters for which the energy of the
magnetosphere interaction with the solar wind plasma has minimum can be found.
At the same time,
the characteristic time scale of the current system response to the changes in the
interplanetary medium is about one hour. Thus, at the present moment the
magnetotail current system tends to realize a state corresponding to the
dynamical pressure value of one hour before:
\begin{equation}
\label{eq5}
\Phi _{0}\left( t\right) =400\cdot \left( p_{sw}\left( t-1\right)%
 \right) ^{1/6.6} .
\end{equation}
When
calculating the dynamical contribution, use was made of the reliable
formula {\it Alexeev et al.} [2001]:
\begin{equation}
\label{eq6}
\Phi _{s}\left( t\right) =b_{t}\left( t\right) \frac{\pi R^{2}_{1}}{2}%
\sqrt{\frac{2R_{1}}{R_{2}}+1} ,
\end{equation}
where \( b_{t}(t) \) is the magnetic field of the magnetotail current system
near the inner edge of the current sheet. In the earlier study
{\it Alexeev et al.} [2001]
this value has been determined using  the
 \( AL \) index. This technique is convenient for comparably weak magnetic
storms (10-12 January, 1997) in the course of which  \( AL \) index
is a good indicator of both auroral activity and the extent of injection
to the ring current. During strong
storms and high substorm activity a direct use of the \( AL \) index to
calculate the contribution of the magnetotail current system is not quite
correct, as it is impossible to determine the energy fractions transferred
through one or another channel during substorm. In this case
only the electric field in the interplanetary medium can be the main measure of energy
accumulation in
the magnetotail:
\begin{equation}
\label{eq7}
b_{t}\left( t\right) =f_{vb}\left( V,B_{imf},t\right) f_{pb}\left(%
p_{sw}\right) f_{ab}\left( AL\right)
\end{equation}
In the above expression the terms $f_{ab}$ and $f_{pb}$ indicate the influence
of substorm activity and solar wind dynamical pressure on the tail current.
The most essential magnetic field dependence on the electric field in
the solar wind is represented by $f_{vb}$ which
is determined by the following expression:
\begin{equation}
\label{eq8}
f_{vb}\left( V,B_{imf},t\right) =\left\{ \begin{array}{ll}
0 & b_{z}\geq 0\\
\,  & \, \\
\frac{V\left( t-1\right) }{300}(b_{z}\left( t-1\right) - & b_{z}<0\\
\left| b_{y}\left( t-1\right) \right| f_{pb}\left( p_{sw}\right) ) &
\end{array}\right.
\end{equation}

In order to explain this formula we should kept in mind that
the current in the current sheet depends mainly  on the term
 \( V\cdot b_{z} \), and one hour delay should be taken into
 acount. Here one hour was obtained for two reasons. Firstly,
this is the minimal time step used when studying magnetic storm and
modeling the \( D_{st} \) variation. Secondly, an  average duration of
substorm is also about one hour. Thus, minute time scale
would be an overestimation of accuracy.


Figure \ref{fig_dts1} represents the measured \( D_{st} \) -
index (see the  solid curve) and the magnetic field at the
Earth's surface given by the Paraboloid model
(see the dot-dot-dashed curve). One can see good agreement of the model calculations
with the real \( D_{st} \). The root mean square deviation is about \( 10 \) nT.
Figure~\ref{fig_dts1} also represents a relative contribution of the large scale
magnetospheric
current systems to \( D_{st} \) (see the figure legend).


\newpage

\vspace{0.2cm}
\textbf{A4. Nonstationary Case. Case studies for June 25--26, 1998 and October 
21-23, 1999 magnetic storms. Comparison with the other models}

The cross-comparison of three
models based on absolutely different principles (A2000, T01, and event
oriented G02 [{\it Ganushkina et al.}, 2002]) was made for the magnetic storms 
on June 25--26, 1998, when $Dst$ reached -120~nT
and one intense storm on October 21-23, 1999, when $Dst$ dropped
to -250~nT. We compared the observed magnetic field from GOES~8,
GOES~9, and GOES~10, Polar and Geotail satellites as well as Dst index with the
magnetic field given by the three models to estimate their
reliability. 

To contrast and to examine the
reliability of the three models, we present here a comparison
of the model results with magnetic measurements from various
spacecraft during the June 1998 and October 1999 storms. We calculate
the magnetic field along the spacecraft orbits located in the
different regions of space: geostationary orbit (GOES-8, -9, and
-10), near-Earth's tail (Geotail), and high-latitude magnetosphere
(Polar). Analysis of simultaneous measurements in the different
magnetospheric regions helps to determine the role of different
magnetospheric current systems during magnetic storms.



\begin{figure}[htbp]
{\par\centering
\resizebox*{8cm}{!}{\includegraphics{./fign/Figure2.eps}}
\par}
\caption{Comparison of the observed $B_x$ and $B_z$ components of the
external magnetic field in the GSM coordinates (thin
lines) with A2000 model results (thick lines) for GOES~8 (two upper
panels), GOES~9 and GOES~10 (next two panels), Polar (next two panels)
and Geotail (bottom two panels) for (a) June 25-26, 1998 and (b) for
October 21-23, 1999 storm events.}
\end{figure}



Figure~18 shows the $B_x$ and $B_z$ components of the external
magnetic field obtained from observations shown by thin lines
and A2000 model results shown by thick lines for GOES~8 (two upper
panels), GOES~9 and GOES~10 (next two panels), Polar (next two panels)
and Geotail (bottom two panels) for (a) June 25-26, 1998 and (b) for
October 21-23, 1999 storm events. Dashed grid lines show the
noon locations for GOES spacecrafts, and perigees of the
Polar orbit. Figures~19 and 20 show the observed and model magnetic
fields in the same format for the event-oriented model G2003 and the
Tsyganenko T01 model, respectively. $B_x$ and $B_z$ measured
components represent the main changes in the magnetospheric current
systems. Their comparisons with the model results reveal
the main model's features.

It can be seen that generally all models show quite good
agreement with observations. For the moderate storm the $B_x$
measured at geosynchronous orbit is better represented by the
A2000
and T01 models whereas the G2003 model gives more accurate
reproduction of the $B_z$ component. The large observed $B_x$
values imply the existence of intense currents that can be
either field-aligned or perpendicular, or an even stronger
compression of the magnetosphere than that represented by the
magnetopause current intensification in the G2003 model. The A2000
model represents the magnetopause size variations depending not
only on solar wind pressure but also on IMF $B_z$ based on {
Shue et al}. (1998) model. The A2000 describes the $B_x$ values
during the
magnetic storm main phase (the first 6 hours of 26 June, 1998) more
accurately than the other models. On the other hand, the A2000 model
underestimates the $B_z$ values during this time interval. This is
because the paraboloid model represents the cross-tail currents as a
discontinuity between the oppositely directed magnetic field bundles in
the southern and northern tail lobes and as a result gives a very small
$B_z$ component in the vicinity of the tail current.

\begin{figure}[htbp]
{\par\centering
\resizebox*{7.5 cm}{!}{\includegraphics{./fign/Figure3.eps}}
\resizebox*{7.5 cm}{!}{\includegraphics{./fign/Figure4.eps}}
\par}
\caption{Observed and model magnetic fields in the same format as
in Figure~18 for the event-oriented model G2003
the empirical T01 models.}
\end{figure}


In general, all three models show approximately similar accuracy
in the representation of magnetic field data observed by Polar.
The G2003 model magnetic field agrees with the observed field at
Geotail (from 0020 UT 25 June until 18 UT 26 June while the
spacecraft was inside the magnetosphere) slightly better than
that given by the A2000 and T01 models.

During the intense storm on October 21-23, 1999 the $B_x$
components from GOES~8 and GOES~10 and Polar are best
represented by the T01 model. At the same time, the T01
model underestimates the $B_z$ component significantly at the
storm maximum. Model $B_z$ values were equal to -230~nT and
-250~nT around 0600 UT on October 22, 1999 while the observed
ones were -50~nT and -80~nT at GOES~8 and GOES~10, respectively.
At that time, GOES~8 was around midnight and GOES~10 was moving
toward midnight in the dusk sector. At the storm maximum, Polar
observations on the duskside showed $B_z$=-25~nT while the T01
model gave $B_z$=-100~nT. Similarly to the moderate storm, the
G2003 model reproduces the $B_z$ variations at GOES and Polar
with enough accuracy.

The local magnetic field variations near the magnetospheric tail
current sheet along the Geotail orbit are not quite correctly
reproduced by the models. The A2000 model gives additional
discrepancies (eg. $B_x$ drops) that arise from the construction
of the tail current model discussed above. However, for both
storm events the $B_x$ components are described with a
reasonable accuracy at GOES~8 and GOES~10 as well as at Polar.

Table~1 shows the RMS deviations  between the satellite measurements and
model calculations determined as $\delta B= \sqrt{\frac{1}{N}
\sum_{i=1}^N ({B_{obs}-B_{model})}^2}$. The obtained discrepancies are
calculated during the whole considered time-intervals and include quiet
as well as disturbed periods. We note that for each orbit the models give
the accuracy of about half of the average value of the magnetic field. In
general, all models represent well the global variations of
magnetospheric magnetic field measured by spacecraft. However, the model
features determine the specific behavior of the magnetic field calculated
in different magnetospheric regions by different models during the
different phases of the considered magnetic storms.


\begin{table}
\caption{The RMS deviations in {nT} between the observed and
modelled magnetic field calculated by the paraboloid (A2000,
Alexeev
et al.,
2001), event-oriented (G2003, Ganushkina et al., 2003), and
Tsyganenko
(T01, Tsyganenko, 2002) models during magnetic storms on 
25-26 June, 1998 and 21-23 October, 1999.}
\begin{tabular}{|l|r|r|r|}
\hline
Satellites & A2000 & G2003 & T01\\
\hline
25-26 June, 1998 \\
\hline
GOES~8   & 18.9& 16.8 &  18.3 \\
GOES~9   & 21.2 & 22.4 & 16.5 \\
Polar   & 26.7& 33.5 &  28.2 \\
Geotail & 28.4& 21.0 &  21.6 \\
\hline
21-23 October, 1999\\
\hline
GOES~8   & 37.0 & 30.2 & 32.1 \\
GOES~10   & 33.7 & 29.4 & 32.7 \\
Polar   & 40.0 & 35.4 & 32.7 \\
Geotail & 22.6 & 11.8 & 11.4 \\
\hline
\end{tabular}
\end{table}

Paraboloid model reproduces well the $B_x$ components of the magnetic
field measured along the GOES and Polar orbits for any level of
disturbances but underestimates $B_z$ depression, due to tail current
model features and possibly due to absence of the partial ring current
model in A2000. The T01 model also provides good agreement between the
observed and modelled $B_x$ component. On the other hand during the
intense storm maximum, the model $B_z$ is significantly  more depressed
than that observed one along the Goes and Polar orbit. Because the ring
current can not give the significant contribution to the magnetic field
at geostationary orbit, we propose that this discrepancy is due to
overestimation of the tail current  contribution. Apparently, this is the
consecuence of the general approach used in development of any empirical
model. Calculation results are very sensitive to the database used for
the model construction. Intense storms are only small part of such
databases. As a result just during extremely disturbed conditions
empirical model demonstrate the sufficient discrepancies.  The
event-oriented model G2003 represents better the substorm-associated
variations of the $B_z$ component at  geosynchronous orbit during both
moderate and intense storms, but gives discrepancies in $B_x$ variation
during storm maximum. 

\begin{figure}[htbp]

{\par\centering
\resizebox*{6cm}{!}{\includegraphics{./fign/Figure6.eps}}
\par}
\caption{Model contributions to $Dst$ and total $Dst$ during June 25-26,
1998 and
October 21-23, 1999 storm events in the same format as in
Figure~5. The quiet-time contributions from the different current
systems are subtracted from the model magnetic field variations.}
\end{figure}

Figure~21 shows the model contributions and total $Dst$ during
June 25-26, 1998 and October 21-23, 1999 storm events.


In general all the models reproduce well Dst and confirm the assumption that
the tail current magnetic field can be sufficiently large to provide a
significant contribution to the $Dst$ variation ({Alexeev et al.}, 1996).
However, the global A2000, G2003 and T01 models demonstrate different tail
current development during magnetic storms. While during the moderate storm the
tail current and ring current have approximately equal maximum contributions to
$Dst$ during the strong magnetic storm the models reveal a different behavior.
The tail current becomes the major contributor to $Dst$ in the T01 model, while
the tail current contribution is smaller than that of the ring current in the
A2000 and G2003 models.


\mbox{ }\vspace{0.5cm}

{\bf Acknowledgments.}
\hspace{0.5cm}
The authors thank N. Tsyganenko NASA GSFC for the magnetosphere magnetic field
database and H. Singer, National Geophysical Data Center (NOAA) for the
GOES data. Wind data were obtained via on-line CDAWeb service operated by
National Space Science Data Center (NASA). The authors thank Tuija Pulkkinen (FMI) for
many useful comments and Natasha Ganushkina (FMI) for the  presenting results of her
model
calculations.

\mbox{ }\vspace{1cm}

{\bf References}
%\begin{references}

Alexeev, I. I., Regular magnetic field in the Earth's
magnetosphere, {\it Geomagn. Aeron., Engl. Transl.,  18}, 447,
1978.

Alexeev, I. I., E. S. Belenkaya, and C.~R.~Clauer,
A model of region 1 field--aligned currents dependent on ionospheric
conductivity and solar wind parameters,  {\it J. Geophys.
Res.,
105}, 21,119, 2000.

Alexeev, I. I., and S. Y. Bobrovnikov, Tail current
sheet
dynamics
during substorm (in Russian), {\it Geomagn. Aeron., 37,} 5, 24, 1997.

Alexeev, I. I., and Y. I. Feldstein,
Modeling of geomagnetic field during magnetic storms and comparison with
observations, {\it J. Atmos. Sol. Terr. Phys., 63}, 331-340, 2001.

Alexeev I.I.,
Kalegaev V.V., Belenkaya E.S., Bobrovnikov S.Yu.,  Feldstein Ya.I.,
Gromova L.I.,
Dynamic model of the magnetosphere: Case study for January 9-12, 1997,
 {\it J. Geophys. Res.}, {\bf 106},  25,683-25,694, 2001.

Burton, R. K., R. L. McPherron, and C. T. Russell,
An empirical
relationship between interplanetary conditions and $Dst$,
{\it J. Geophys. Res.}, {\bf 80}, 4204-4213, 1975.

 Clauer C. R. Jr., I. I. Alexeev, E. S. Belenkaya, and J. B. Baker,
Special
features of the September 24-27, 1998 storm during high solar wind dynamic
pressure and northward interplanetary magnetic field,
{\it J. Geophys. Res., 106},  25,695-25,712, 2001.

Dessler, A. J., and E. N. Parker,  Hydromagnetic theory of
geomagnetic
storms, {\it J.~Geophys. Res., 64}, 2239-2252, 1959.

Faierfield  et al., A large magnetosphere magnetic field database,
{\it J. Geophys.
Res.,
99}, 11,319,
1994.
Feldstein, Y.I., et al.,
% L.I. Gromova, A. Grafe, C.-I. Meng, V.V. Kalegaev, I.I. Alexeev, Yu.P. Sumaruk:
Dynamics of the auroral elecrtojets and their mapping to the
magnetosphere',
 {\it Ragiation Measurements, 30}, N 5,1999.
579-587.

Ganushkina, N. Y., Pulkkinen, T. I., Kubyshkina, M. V., Singer, H.
J.,  Russell, C. T., Modeling the ring current magnetic field during storms,
{\it J. Geophys. Res., 107}, SMP 3-1 to SMP 3-13, 2002.


Mead, G.D., and D.H. Faierfield, A quantitative magnetospheric model derived
from spacecraft magnetometer data, {\it J. Geophys. Res., 80,} 523, 1975.


O'Brien,~T.~P., and R.~L.~McPherron,
An empirical phase space
analysis of ring current dynamics: Solar wind control of injection
and decay, {\it J. Geophys. Res., 105},  7707--7719, 2000.


Olson, W.P., and K.A. Pfitzer, A quantitative model of the magnetospheric
magnetic field, {\it J. Geophys. Res., 79,} 3739, 1974.

Reeves et al.,  The relativistic electron response at geosynchronous
orbit
during the January 1997 magnetic storm, {\it J. Geophys. Res., 103,}
17,559, 1998.

Shue, J.-H., J. K. Chao, H. C. Fu, C. T.
Russell, P. Song, K. K. Khurana, and H. J. Singer, A new
functional form to study the solar wind control of the
magnetopause size and shape, { J. Geophys. Res.,
102,} 9497--9512, 1997.



Tsyganenko, N.A., Global quantitative models of the geomagnetic field in the
cislunar magnetosphere for different disturbance levels, Planet.Space Sci.,
35, 1347-1358, 1987.


Tsyganenko, N.A., A magnetospheric magnetic field model with the warped
tail current sheet, Planet. Space Sci., 37, 5-20, 1989.

Tsyganenko, N.A., Modeling the Earth's magnetospheric
magnetic
field confined within a realistic magnetopause, {\it J. Geophys.
Res., 100,}
5599, 1995.



Tsyganenko, N. A., A model of the near magnetosphere with a
dawn-dusk asymmetry: 1. Mathematical structure, {\it  J. Geophys.
Res.}, { 107,} 1179, 10.1029/2001JA0002192001, 2002a.



Tsyganenko, N.A.,  A model of the near magnetosphere
with a dawn-dusk asymmetry: 2. Parameterization and
fitting to observations, {\it  J. Geophys. Res., 107},
1176,  10.1029/2001JA900120, 2002b.


Turner,~N.~E., D.~N.~Baker, T.~I.~Pulkkinen, and R.~L.~McPherron,
Evaluation of the tail current contribution to $Dst$, {\it J.
Geophys. Res.,
105,} 5431, 2000..

%\end{references}

\end{document}

